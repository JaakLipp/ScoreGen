\begin{enumerate}
  \item What went well while writing this deliverable? 
  \textbf{Mark} This deliverable was a good opportunity to coordinate our efforts amongst a much larger body of work compared to the first deliverable. By maintaining a continuous line of feedback and communication, our team helped avoid contradictions within the documentation, and sections such as “stretch goals”, “future plans”, “waiting room” etc were very helpful in having the group think forward and define exactly what was happening for sure vs what might happen provided things go well.\\ \\
  \textbf{Ian}  I think we communicated and split work well during this deliverable. In some ways this deliverable acted as the official foray into the capstone project because it was much larger than the previous deliverables. It also required that we further narrow down our vision for the project which I think excited us all.  \\ \\
  \textbf{Emily} This deliverable really helped us to gain a better understanding of the different aspects of this project we’ll be taking on. It can be difficult to have a thorough understanding of a project as large as this one from the get go, and I think that we worked really well as a team to communicate and prioritise the features we’d like to see in the final product.\\ \\
  \textbf{Jackson}  I thought this was great because the deliverable cemented how we will work as a team. I thought we communicated very well on each section, referencing each other’s work and asking for clarification when needed. I also enjoyed our open discussions when we were discussing what we would be implementing in the future, as thinking ahead is always good for these projects \\ \\
  \item What pain points did you experience during this deliverable, and how did
  you resolve them?
  \textbf{Mark} A pain point of the deliverable was trying to minimise repetition in similar sections of the requirements document. It often felt as though the same thing was being said multiple times throughout the different sections, and it was difficult to maintain a cohesive flow among sections.\\ \\ 
  \textbf{Ian} One of the pain points I encountered was trying to apply the Volere template (specifically business data and adjacent business analysis) to a project that is more or less standalone. The template is definitely geared towards very complex projects that involve making additions to existing systems and interacting with many other organisations. While our project does do this in some sense, it was difficult to draw parallels when using the template and its descriptions of the sections.\\ \\
  \textbf{Emily} The big pain point I ran into was trying to differentiate what a functional vs non-functional requirement should look like. This is something I’ve had trouble with since I first started writing requirements, and I found myself going back and forth between which category a requirement belonged to. I ended up just doing a lot of research on how to write a good requirement, and I found that having examples to compare this project against was what helped me gain clarity in this process.\\ \\
  \textbf{Jackson} There was one main pain point I experienced when doing this deliverable, which was dealing with non-functional vs functional requirements. This has always been a tough thing for me to rationalise in my head since sometimes they can be very close to each other. In the case of this deliverable, I was dealing with the performance non-functional requirements which made me second guess myself since they sounded a lot like functional requirements. I resolved that by learning more about the differences between the two, and how non-functional requirements can deal with the performance of the system whereas functional requirements deal with how the system must work. \\ \\
  \item How many of your requirements were inspired by speaking to your
  client(s) or their proxies (e.g. your peers, stakeholders, potential users)? \\
  The majority of requirements were derived through personal experience and understanding of necessary features for an acceptable product. However, requirements related to user proficiency in music theory were formed by communicating with potential clients with varying levels of music understanding. This helped to determine what may be assumed to be common knowledge, versus what may require additional documents for users to understand \\
  \item Which of the courses you have taken, or are currently taking, will help
  your team to be successful with your capstone project.
  \textbf{Mark} In terms of technical knowledge, the most valuable courses will have been 3MX3 and 3DX4, both focused on signal processing, a key aspect of the project. Additionally, experiential courses like 3XB3 and 2AA4 provided a strong foundation in building Git-based projects and collaborating on code within a team. \\ \\
  \textbf{Ian} 3MX3 has to be the course that relates most to this project. This course provides valuable insight into the world of signal processing which is arguably the most important part of the project. Other than that, 4AA4 has given some valuable knowledge about real-time tasks and scheduling which will be useful depending on the direction we take with the goals of the project. 4HC3 also fits very well into this project as it gave me experience not only in conducting stakeholder/client analysis but also in designing interfaces that put a large focus on user experience. \\ \\
  \textbf{Emily} At this point in the course, the largest similarities I’ve seen have been with SFWRENG 3A04, Large System Design. This course was set up very similarly to capstone, as most of the marks received came from a term-long project with a large focus on requirements elicitation, use cases, and stakeholder definitions. Going back to my notes from 3A04 helped me a lot for this deliverable, and will likely continue to be a main source of truth moving forward.\\ \\
  \textbf{Jackson} A large part of this project will be signal processing, for which I have taken 3MX3 and 3DX4 which directly deal with the subject matter that we’ll need to use to process the signals. Additionally, all of the assignment-based courses (e.g. 2aa4, 3bb4) gave me the experience to feel confident when starting these larger projects. \\ \\
  \item What knowledge and skills will the team collectively need to acquire to
  successfully complete this capstone project?  Examples of possible knowledge
  to acquire include domain specific knowledge from the domain of your
  application, or software engineering knowledge, mechatronics knowledge or
  computer science knowledge.  Skills may be related to technology, or writing,
  or presentation, or team management, etc.  You should look to identify at
  least one item for each team member. \\
  Successful completion of this capstone project will be reliant on the team acquiring both technical and non-technical skills. Project management has already proven to be a vital part of this project, requiring leadership, an understanding of agile methodologies, and an ability to delegate work where necessary. Music theory and composition knowledge is the entire basis of our problem statement, and will be a necessary skill to ensure our system is fulfilling its purpose and providing accurate transcriptions. UI/UX design will need knowledge of graphic design, the qualities of a good user interface, and prototyping tools such as Figma. Software testing and quality analysis skills will be exercised through use of our Kanban board and ensuring we have a robust suite of regression tests in our CI pipeline. Signal processing and related controls skills will also be an important part of processing audio input from the user. Given these 5 knowledge areas, each member of our group can choose one to laser focus on and ensure our breadth of knowledge sufficiently covers the topics needed in our project's success.\\
  \item For each of the knowledge areas and skills identified in the previous
  question, what are at least two approaches to acquiring the knowledge or
  mastering the skill?  Of the identified approaches, which will each team
  member pursue, and why did they make this choice? \\
  For all of the knowledge areas mentioned above, there are two approaches that we can apply in the context of this project. The first is self-directed learning with online resources. There are plenty of free courses, tutorials, tools, and texts online that the team can take advantage of. The second approach is to develop and practise these skills with small personal projects, experiments, or even by contributing to open source projects that involve the skills. Both of these approaches are reasonable given that the team is composed of full-time students and because both cater well to self-directed and self-paced learning. Due to the breadth of knowledge and skills required by the project, the team will collaboratively pursue both of these approaches. Therefore, individually, each team member will pursue both simultaneously sharing newly gained skills with the team where possible. This choice is a result of the desire for the team to increase collaboration and to tackle the knowledge gaps we may have as efficiently as possible. 
\end{enumerate}
