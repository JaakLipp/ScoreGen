\documentclass{article}

\usepackage{float}
\restylefloat{table}

\usepackage{booktabs}

\title{Team Contributions: POC\\\progname}

\author{\authname}

\date{}

%% Comments

\usepackage{color}

\newif\ifcomments\commentstrue %displays comments
%\newif\ifcomments\commentsfalse %so that comments do not display

\ifcomments
\newcommand{\authornote}[3]{\textcolor{#1}{[#3 ---#2]}}
\newcommand{\todo}[1]{\textcolor{red}{[TODO: #1]}}
\else
\newcommand{\authornote}[3]{}
\newcommand{\todo}[1]{}
\fi

\newcommand{\wss}[1]{\authornote{blue}{SS}{#1}} 
\newcommand{\plt}[1]{\authornote{magenta}{TPLT}{#1}} %For explanation of the template
\newcommand{\an}[1]{\authornote{cyan}{Author}{#1}}

%% Common Parts

\newcommand{\progname}{ProgName} % PUT YOUR PROGRAM NAME HERE
\newcommand{\authname}{Team \#, Team Name
\\ Student 1 name
\\ Student 2 name
\\ Student 3 name
\\ Student 4 name} % AUTHOR NAMES                  

\usepackage{hyperref}
    \hypersetup{colorlinks=true, linkcolor=blue, citecolor=blue, filecolor=blue,
                urlcolor=blue, unicode=false}
    \urlstyle{same}
                                


\begin{document}

\maketitle

This document summarizes the contributions of each team member up to the POC
Demo.  The time period of interest is the time between the beginning of the term
and the POC demo.

\section{Demo Plans}

The Proof of Concept will be a demonstration of the team’s ability to develop the 
most crucial aspect of the final product, which is to take in audio input and output 
the notes’ pitches and lengths. This will be achieved by using a prepared audio file (in 
.wav or .mp4 format). The input signals will then be processed to identify note pitch, rhythm, 
and timing. Although the final product will include sheet music generation, this will not be 
demonstrated. The focus of the demonstration will lie with the proper transcription of notes that 
are input into the system through a pre-made audio file. Differentiating between similar pitches 
such as A\# and Bb could be potentially difficult tasks. The Proof of Concept will help dispel concerns 
by allowing the team to show the product's potential for accurate signal processing. Given the ability 
to accurately process audio, the completion of the product will be simplified, as the focus can 
shift towards the construction of a user interface and adding features while optimizing performance and accuracy.

\section{Team Meeting Attendance}

\begin{table}[H]
\centering
\begin{tabular}{ll}
\toprule
\textbf{Student} & \textbf{Meetings}\\
\midrule
Total & 7\\
Jackson Lippert & 7\\
Emily Perica & 7\\
Ian Algenio & 7\\
Mark Kogan & 7\\
\bottomrule
\end{tabular}
\end{table}

\section{Supervisor/Stakeholder Meeting Attendance}

\begin{table}[H]
\centering
\begin{tabular}{ll}
\toprule
\textbf{Student} & \textbf{Meetings}\\
\midrule
Total & 1\\
Jackson Lippert & 1\\
Emily Perica & 1\\
Ian Algenio & 1\\
Mark Kogan & 1\\
\bottomrule
\end{tabular}
\end{table}

\section{Lecture Attendance}

\begin{table}[H]
\centering
\begin{tabular}{ll}
\toprule
\textbf{Student} & \textbf{Lectures}\\
\midrule
Total & 6\\
Jackson Lippert & 1\\
Emily Perica & 4\\
Ian Algenio & 5\\
Mark Kogan & 5\\
\bottomrule
\end{tabular}
\end{table}

5 lecture issues in our kanban board. 

\section{TA Document Discussion Attendance}

\begin{table}[H]
\centering
\begin{tabular}{ll}
\toprule
\textbf{Student} & \textbf{Lectures}\\
\midrule
Total & 3\\
Jackson Lippert & 3\\
Emily Perica & 3\\
Ian Algenio & 2\\
Mark Kogan & 3\\
\bottomrule
\end{tabular}
\end{table}

Ian was not in Hamilton during the Oct 28 TA meeting.

\section{Commits}

\begin{table}[H]
\centering
\begin{tabular}{lll}
\toprule
\textbf{Student} & \textbf{Commits} & \textbf{Percent}\\
\midrule
Total & 65 & 100\% \\
Jackson Lippert & 17 & 26\% \\
Emily Perica & 24 & 37\% \\
Ian Algenio & 15 & 23\% \\
Mark Kogan & 17 & 26\% \\
\bottomrule
\end{tabular}
\end{table}

\section{Issue Tracker}

\begin{table}[H]
\centering
\begin{tabular}{lll}
\toprule
\textbf{Student} & \textbf{Authored (O+C)} & \textbf{Assigned (C only)}\\
\midrule
Jackson Lippert & 0 & 12\\
Emily Perica & 61 & 12\\
Ian Algenio & 1 & 12\\
Mark Kogan & 2 &10\\
\bottomrule
\end{tabular}
\end{table}

\section{CICD}

As stated in the \href{https://github.com/emilyperica/ScoreGen/blob/main/docs/DevelopmentPlan/DevelopmentPlan.pdf}{Development Plan document}:
CI/CD Workflow Will include the following
Actions on Push/Pull Request:
\begin{itemize}
    \item LaTeX to PDF converter
    \item PR naming convention enforcer
    \item Code packaged as an executable (To do)
    \item Linting (To do)
    \item Unit tests (To do)
\end{itemize}

Note that a PR may not be merged if any failures are present in the pipeline.

\end{document}