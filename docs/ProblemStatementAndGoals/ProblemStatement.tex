\documentclass{article}

\usepackage{tabularx}
\usepackage{booktabs}

\title{Problem Statement and Goals\\\progname}

\author{\authname}

\date{}

%% Comments

\usepackage{color}

\newif\ifcomments\commentstrue %displays comments
%\newif\ifcomments\commentsfalse %so that comments do not display

\ifcomments
\newcommand{\authornote}[3]{\textcolor{#1}{[#3 ---#2]}}
\newcommand{\todo}[1]{\textcolor{red}{[TODO: #1]}}
\else
\newcommand{\authornote}[3]{}
\newcommand{\todo}[1]{}
\fi

\newcommand{\wss}[1]{\authornote{blue}{SS}{#1}} 
\newcommand{\plt}[1]{\authornote{magenta}{TPLT}{#1}} %For explanation of the template
\newcommand{\an}[1]{\authornote{cyan}{Author}{#1}}

%% Common Parts

\newcommand{\progname}{ProgName} % PUT YOUR PROGRAM NAME HERE
\newcommand{\authname}{Team \#, Team Name
\\ Student 1 name
\\ Student 2 name
\\ Student 3 name
\\ Student 4 name} % AUTHOR NAMES                  

\usepackage{hyperref}
    \hypersetup{colorlinks=true, linkcolor=blue, citecolor=blue, filecolor=blue,
                urlcolor=blue, unicode=false}
    \urlstyle{same}
                                


\begin{document}

\maketitle

\begin{table}[hp]
\caption{Revision History} \label{TblRevisionHistory}
\begin{tabularx}{\textwidth}{llX}
\toprule
\textbf{Date} & \textbf{Developer(s)} & \textbf{Change}\\
\midrule
Date1 & Name(s) & Description of changes\\
Date2 & Name(s) & Description of changes\\
... & ... & ...\\
\bottomrule
\end{tabularx}
\end{table}

\section{Problem Statement}

\subsection{Problem}
Musicians often improvise or develop complex pieces, but have difficulty sharing, expanding on, or documenting their progress. Manual note-taking is also time-consuming and prone to errors.

\subsection{Inputs and Outputs}

The product aims to take in audio from a singular musical instrument, and after processing the data, output the sheet music representative of the audio provided. The sheet music will contain the notes, note duration, and overall time signature of the piece.


\subsection{Stakeholders}

The product is largely aimed at helping aspiring musicians who either lack the time or the theory background to document their work in a tangible written form. It may be useful for collaborating artists who need a quick way to communicate suggestions or potential melodies.

\subsection{Environment}
To use this product the user simply has to download the desktop application associated with the product, and a high-quality microphone attachment to their computer

\section{Goals}
\begin{enumerate}
    \item The resulting product can handle monophonic and polyphonic input from a 
    musical instrument.
    \begin{itemize}
        \item The core function of the product.
        \item Elements such as note pitch, note length, rhythm, and tempo are essential 
        for creating sheet music.
    \end{itemize}
    \item The resulting product can generate a readable and usable score that reasonably matches the 
    audio input.
        \begin{itemize}
            \item These are the minimum qualities such that the score can be played to recreate the 
            original input.
        \end{itemize}
    \item The resulting product's digitized audio and generated scores are exportable to standard file 
    formats.
    \begin{itemize}
        \item To improve portability and user accessibility.
        \item Maximizes the types of devices a user may use to read their transcriptions.
    \end{itemize}
    \item The resulting product provides a graphical user interface (GUI) that adheres to 
    human-centred design principles.
    \begin{itemize}
        \item Puts the user needs, capabilities, and behaviours first, which helps achieve 
        usability goals such as learnability.
        \item Enhances overall user experience.
    \end{itemize}
    \item The resulting product is downloadable and easily installed by the user.
    \begin{itemize}
        \item Essential for usability.
        \item Broadens the number of users that the product can reach. A product that is difficult
        to install easily deters users.
    \end{itemize}
\end{enumerate}

The ultimate goal is to develop a fast, accurate sheet music generator paired with an intuitive, user-friendly interface that requires minimal effort to learn.

The ultimate goal is to develop a fast, accurate sheet music generator paired with an intuitive, user-friendly interface that requires minimal effort to learn.

\section{Stretch Goals}
\begin{enumerate}
    \item Available on different platforms and devices.
    \item Allows for real-time transcription to provide user feedback while they are playing an
    instrument.
    \item Detailed and improved transcription by expanding musical context and
    structure identification (elements such as dynamics, expression, modulations, etc.).
    \item Provides optimal instrument fingering tablature for fretted stringed instruments.
    \item The resulting product is available without an internet connection.
\end{enumerate}

\section{Challenge Level and Extras}
A general challenge level is expected for this project. The rationale behind this selected level is due to
the limited domain knowledge required to successfully achieve the goals of this project. There are two main 
aspects of knowledge needed: music theory and signals and systems. Much of the necessary knowledge of the 
latter topic has been discussed in multiple junior level undergraduate courses and the former is either already
known by multiple members of the team or can easily be researched and understood.
As a result of the expected challenge level, two extras have been chosen:
\begin{enumerate}
    \item Design thinking.
    \item Usability testing.
\end{enumerate}

\newpage{}

\section*{Appendix --- Reflection}

\wss{Not required for CAS 741}

The purpose of reflection questions is to give you a chance to assess your own
learning and that of your group as a whole, and to find ways to improve in the
future. Reflection is an important part of the learning process.  Reflection is
also an essential component of a successful software development process.  

Reflections are most interesting and useful when they're honest, even if the
stories they tell are imperfect. You will be marked based on your depth of
thought and analysis, and not based on the content of the reflections
themselves. Thus, for full marks we encourage you to answer openly and honestly
and to avoid simply writing ``what you think the evaluator wants to hear.''

Please answer the following questions.  Some questions can be answered on the
team level, but where appropriate, each team member should write their own
response:


\begin{enumerate}
    \item What went well while writing this deliverable? 
    \item What pain points did you experience during this deliverable, and how
    did you resolve them?
    \item How did you and your team adjust the scope of your goals to ensure
    they are suitable for a Capstone project (not overly ambitious but also of
    appropriate complexity for a senior design project)?
\end{enumerate}  

\end{document}