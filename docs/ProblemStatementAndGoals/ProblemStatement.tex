\documentclass{article}

\usepackage{tabularx}
\usepackage{booktabs}

\title{Problem Statement and Goals\\\progname}

\author{\authname}

\date{}

%% Comments

\usepackage{color}

\newif\ifcomments\commentstrue %displays comments
%\newif\ifcomments\commentsfalse %so that comments do not display

\ifcomments
\newcommand{\authornote}[3]{\textcolor{#1}{[#3 ---#2]}}
\newcommand{\todo}[1]{\textcolor{red}{[TODO: #1]}}
\else
\newcommand{\authornote}[3]{}
\newcommand{\todo}[1]{}
\fi

\newcommand{\wss}[1]{\authornote{blue}{SS}{#1}} 
\newcommand{\plt}[1]{\authornote{magenta}{TPLT}{#1}} %For explanation of the template
\newcommand{\an}[1]{\authornote{cyan}{Author}{#1}}

%% Common Parts

\newcommand{\progname}{ProgName} % PUT YOUR PROGRAM NAME HERE
\newcommand{\authname}{Team \#, Team Name
\\ Student 1 name
\\ Student 2 name
\\ Student 3 name
\\ Student 4 name} % AUTHOR NAMES                  

\usepackage{hyperref}
    \hypersetup{colorlinks=true, linkcolor=blue, citecolor=blue, filecolor=blue,
                urlcolor=blue, unicode=false}
    \urlstyle{same}
                                


\begin{document}

\maketitle

\begin{table}[hp]
\caption{Revision History} \label{TblRevisionHistory}
\begin{tabularx}{\textwidth}{llX}
\toprule
\textbf{Date} & \textbf{Developer(s)} & \textbf{Change}\\
\midrule
2024/09/24 & Mark, Ian, Emily, Jackson & Initial Revision\\
2025/03/26 & Ian & Update extras, \href{https://github.com/emilyperica/ScoreGen/issues/300}{Issue \#300}\\
2025/04/02 & Ian & Change extra to research report, \href{https://github.com/emilyperica/ScoreGen/issues/355}{Issue \#355}\\
\bottomrule
\end{tabularx}
\end{table}

\section{Problem Statement}

\subsection{Problem}
Musicians often improvise or develop complex pieces, but have difficulty sharing, expanding on, or documenting their progress. Manual note-taking is also time-consuming and prone to errors.

\subsection{Inputs and Outputs}

The product aims to take in audio from a singular musical instrument, and after processing the data, output the sheet music representative of the audio provided. The sheet music will contain the notes, note duration, and overall time signature of the piece.


\subsection{Stakeholders}

The product is largely aimed at helping aspiring musicians who either lack the time or the theory background to document their work in a tangible written form. It may be useful for collaborating artists who need a quick way to communicate suggestions or potential melodies.

\subsection{Environment}
To use this product the user simply has to download the desktop application associated with the product, and a high-quality microphone attachment to their computer

\section{Goals}
\begin{enumerate}
    \item The resulting product can handle monophonic and polyphonic input from a 
    musical instrument.
    \begin{itemize}
        \item The core function of the product.
        \item Elements such as note pitch, note length, rhythm, and tempo are essential 
        for creating sheet music.
    \end{itemize}
    \item The resulting product can generate a readable and usable score that reasonably matches the 
    audio input.
        \begin{itemize}
            \item These are the minimum qualities such that the score can be played to recreate the 
            original input.
        \end{itemize}
    \item The resulting product's digitized audio and generated scores are exportable to standard file 
    formats.
    \begin{itemize}
        \item To improve portability and user accessibility.
        \item Maximizes the types of devices a user may use to read their transcriptions.
    \end{itemize}
    \item The resulting product provides a graphical user interface (GUI) that adheres to 
    human-centred design principles.
    \begin{itemize}
        \item Puts the user needs, capabilities, and behaviours first, which helps achieve 
        usability goals such as learnability.
        \item Enhances overall user experience.
    \end{itemize}
    \item The resulting product is downloadable and easily installed by the user.
    \begin{itemize}
        \item Essential for usability.
        \item Broadens the number of users that the product can reach. A product that is difficult
        to install easily deters users.
    \end{itemize}
\end{enumerate}

The ultimate goal is to develop a fast, accurate sheet music generator paired with an intuitive, user-friendly interface that requires minimal effort to learn.

The ultimate goal is to develop a fast, accurate sheet music generator paired with an intuitive, user-friendly interface that requires minimal effort to learn.

\section{Stretch Goals}
\begin{enumerate}
    \item Available on different platforms and devices.
    \item Allows for real-time transcription to provide user feedback while they are playing an
    instrument.
    \item Detailed and improved transcription by expanding musical context and
    structure identification (elements such as dynamics, expression, modulations, etc.).
    \item Provides optimal instrument fingering tablature for fretted stringed instruments.
    \item The resulting product is available without an internet connection.
\end{enumerate}

\section{Challenge Level and Extras}
A general challenge level is expected for this project. The rationale behind this selected level is due to
the limited domain knowledge required to successfully achieve the goals of this project. There are two main 
aspects of knowledge needed: music theory and signals and systems. Much of the necessary knowledge of the 
latter topic has been discussed in multiple junior level undergraduate courses and the former is either already
known by multiple members of the team or can easily be researched and understood.
As a result of the expected challenge level, two extras have been chosen:
\begin{enumerate}
    \item GenderMag evaluation.
    \item Literature review.
\end{enumerate}

\newpage{}

\section*{Appendix --- Reflection}

\wss{Not required for CAS 741}

The purpose of reflection questions is to give you a chance to assess your own
learning and that of your group as a whole, and to find ways to improve in the
future. Reflection is an important part of the learning process.  Reflection is
also an essential component of a successful software development process.  

Reflections are most interesting and useful when they're honest, even if the
stories they tell are imperfect. You will be marked based on your depth of
thought and analysis, and not based on the content of the reflections
themselves. Thus, for full marks we encourage you to answer openly and honestly
and to avoid simply writing ``what you think the evaluator wants to hear.''

Please answer the following questions.  Some questions can be answered on the
team level, but where appropriate, each team member should write their own
response:


\begin{enumerate}
    \item What went well while writing this deliverable? \\
    \textbf{Emily:} While writing this deliverable we were able to solidify our understanding of the scope of our project, and how we can best leverage it to 
                    gain specific skills we’d like to have for future careers in industry. I think we did a great job of introducing the right amount of 
                    complexity to the project, without being overly ambitious in what we’ll be able to get done.\\ \\
    \textbf{Ian:} I was able to get a good idea of what was expected for the sections. I viewed some past capstone projects and their goals. The software engineering 
                    lecture on 21/07/2024 also provided insight with some examples that we talked through in-class. This was especially helpful when determining the right amount of 
                    abstraction required for the project goals. Overall everything went well, we all love the idea of the project, its scope, and believe we can achieve the goals outlined.\\ \\
    \textbf{Jackson:} In my opinion, almost everything went well in writing this deliverable. I think this is due to our meeting in which we divided up the work 
                    as evenly as possible so that all group members shared the load. This made it possible to each specialize in one area of the development plan, allowing us to 
                    be more in-depth and complete when writing the initial draft.\\ \\
    \textbf{Mark:} I think deliverable went very well overall. It was a great opportunity to stamp out any confusion that I or my partners had about the project, and gave a better overall clarity
                   of how we work together as a team, how we will communicate, and deal with conflict. Overall, it was a relatively simple set of tasks, but gave a good trial run for our team to cooperate, 
                   which I hope will set us up for great success moving forward.\\ \\

    \item What pain points did you experience during this deliverable, and how
    did you resolve them?\\
    \textbf{Emily:} One main pain point that we experienced was using GitHub for collaboration on the document, since we’re all used to being able to see each 
                    other’s edits in real time using either google docs or Microsoft Word. This led to a bit of a disjointed feeling between the individual work 
                    that had been taken on. It ultimately led to us increasing collaboration and overall communication, and we discovered that our group meetings 
                    were vital for hosting brainstorming sessions and getting us all on the same page. \\ \\
    \textbf{Ian:} A major pain point for me was attempting to merge my PRs after editing the .tex template documents. I attempted to resolve the issue and .pdf conflicts multiple times. 
                  I decided to wait until the next team meeting to ask for help from my team. The issue seemed to happen to more than just me, so we agreed to a temporary solution to delete and re-create 
                  the PRs. A more long term solution will be implemented via GHA. Discussing issues like this, when I believe it’s just me struggling with it, is not typical for me. I usually want to solve 
                  it on my own but asking for help proved to be beneficial for not only me but others as well. This served as a reminder that asking for help is never a bad thing.\\ \\
    \textbf{Jackson:} There was one main pain point that sticks out to me for this deliverable which is the generation of PDF files from the LaTeX code. This posed 
                    a problem when we went to commit the PDFs to the repository, as we were prompted to manage merge conflicts with a file that was deemed ‘too complex’. As a result, 
                    we created a CI job to re-generate PDFs based on merges for our LaTeX files, resolving our issue and removing the need to directly commit PDF files.\\ \\
    \textbf{Mark:} A pain point of this project was the necessity for us to all work on the same tex files. This led to merge conflicts since github cannot differentiate line by line for binary files like pdfs. 
                   This was a little annoying because I had to rebase a couple times to ensure my PR could be merged even though I’d be editing different sections from my team members. 
                   This likely won’t be a big issue in the future, as individual partners will have entire tex files assigned to themselves as the project moves forward, but in any case, 
                   we will create a github action to render pdfs on pushes, which will make life easier.\\ \\

    \item How did you and your team adjust the scope of your goals to ensure
    they are suitable for a Capstone project (not overly ambitious but also of
    appropriate complexity for a senior design project)?

    There were three main ways our group adjusted the scope of our goals. The first two ways involved discussing the project with our supervisor Dr. Martin v. Mohrenschildt and Dr. Smith. 
    Although these conversations occurred before composing the problem statement and goals document, both offered valuable insights into the feasibility of our project. Some of our goals were a little too ambitious considering the duration of the course. 
    An example of how these conversations helped narrow down our goals is with our stretch goal of providing real-time transcription of audio input. Initially we believed this to be of appropriate complexity, 
    but decided otherwise when we started to flesh out our project and discovered implementing real-time applications would be difficult with the operating systems the majority of our computers had. 
    Our team also referenced material available in the GitLab repository, mainly past capstone projects. Analyzing the abstraction and complexity of these goals helped provide a framework for our own goals.
\end{enumerate}  

\end{document}