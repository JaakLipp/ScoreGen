\documentclass[12pt, titlepage]{article}

\usepackage{amsmath, mathtools}

\usepackage[round]{natbib}
\usepackage{amsfonts}
\usepackage{amssymb}
\usepackage{graphicx}
\usepackage{colortbl}
\usepackage{xr}
\usepackage{hyperref}
\usepackage{longtable}
\usepackage{xfrac}
\usepackage{tabularx}
\usepackage{float}
\usepackage{siunitx}
\usepackage{booktabs}
\usepackage{multirow}
\usepackage[section]{placeins}
\usepackage{caption}
\usepackage{fullpage}
\usepackage{xr}
\usepackage{hyperref}
\externaldocument{../SoftArchitecture/MG}

\hypersetup{
bookmarks=true,     % show bookmarks bar?
colorlinks=true,       % false: boxed links; true: colored links
linkcolor=red,          % color of internal links (change box color with linkbordercolor)
citecolor=blue,      % color of links to bibliography
filecolor=magenta,  % color of file links
urlcolor=cyan          % color of external links
}

\usepackage{array}

\externaldocument{../../SRS/SRS}

%% Comments

\usepackage{color}

\newif\ifcomments\commentstrue %displays comments
%\newif\ifcomments\commentsfalse %so that comments do not display

\ifcomments
\newcommand{\authornote}[3]{\textcolor{#1}{[#3 ---#2]}}
\newcommand{\todo}[1]{\textcolor{red}{[TODO: #1]}}
\else
\newcommand{\authornote}[3]{}
\newcommand{\todo}[1]{}
\fi

\newcommand{\wss}[1]{\authornote{blue}{SS}{#1}} 
\newcommand{\plt}[1]{\authornote{magenta}{TPLT}{#1}} %For explanation of the template
\newcommand{\an}[1]{\authornote{cyan}{Author}{#1}}

%% Common Parts

\newcommand{\progname}{ProgName} % PUT YOUR PROGRAM NAME HERE
\newcommand{\authname}{Team \#, Team Name
\\ Student 1 name
\\ Student 2 name
\\ Student 3 name
\\ Student 4 name} % AUTHOR NAMES                  

\usepackage{hyperref}
    \hypersetup{colorlinks=true, linkcolor=blue, citecolor=blue, filecolor=blue,
                urlcolor=blue, unicode=false}
    \urlstyle{same}
                                


\begin{document}

\title{Module Interface Specification for \progname{}}

\author{\authname}

\date{\today}

\maketitle

\pagenumbering{roman}

\section{Revision History}

\begin{tabularx}{\textwidth}{p{3cm}p{2cm}X}
\toprule {\bf Date} & {\bf Version} & {\bf Notes}\\
\midrule
Date 1 & 1.0 & Notes\\
Date 2 & 1.1 & Notes\\
\bottomrule
\end{tabularx}

~\newpage

\section{Symbols, Abbreviations and Acronyms}

\begin{itemize}
    \item \textbf{SRS}: System Requirements Specifications
    \item \textbf{MG}: Module Guide
    \item \textbf{M1}: Hardware-Hiding Module
    \item \textbf{M2}: User Interface Module
    \item \textbf{M3}: Score Generation Module
    \item \textbf{M4}: Raw Signal Processing
    \item \textbf{M5}: Audio Feature Extraction (note, key, sig, etc.)
    \item \textbf{M6}: File Format Conversions Module
    \item \textbf{M7}: Audio Recording and Playback Module
\end{itemize}

\newpage

\tableofcontents

\newpage

\pagenumbering{arabic}

\section{Introduction}
This document details the Module Interface Specifications for ScoreGen. A service designed to transcribe user-recorded musical compositions into accurate sheet music by determining pitch, duration, tempo, and more advanced musical features. The service also aims to provide a user interface to use and interact with the product. Complementary documents include the SRS and MG documents. The full documentation can be found on \href{https://github.com/emilyperica/ScoreGen}{github}.


\section{Notation}

\wss{You should describe your notation.  You can use what is below as
  a starting point.}

The structure of the MIS for modules comes from \citet{HoffmanAndStrooper1995},
with the addition that template modules have been adapted from
\cite{GhezziEtAl2003}.  The mathematical notation comes from Chapter 3 of
\citet{HoffmanAndStrooper1995}.  For instance, the symbol := is used for a
multiple assignment statement and conditional rules follow the form $(c_1
\Rightarrow r_1 | c_2 \Rightarrow r_2 | ... | c_n \Rightarrow r_n )$.

The following table summarizes the primitive data types used by \progname. 

\begin{center}
\renewcommand{\arraystretch}{1.2}
\noindent 
\begin{tabular}{l l p{7.5cm}} 
\toprule 
\textbf{Data Type} & \textbf{Notation} & \textbf{Description}\\ 
\midrule
character & char & a single symbol or digit\\
integer & $\mathbb{Z}$ & a number without a fractional component in (-$\infty$, $\infty$) \\
natural number & $\mathbb{N}$ & a number without a fractional component in [1, $\infty$) \\
real & $\mathbb{R}$ & any number in (-$\infty$, $\infty$)\\
\bottomrule
\end{tabular} 
\end{center}

\noindent
The specification of \progname \ uses some derived data types: sequences, strings, and
tuples. Sequences are lists filled with elements of the same data type. Strings
are sequences of characters. Tuples contain a list of values, potentially of
different types. In addition, \progname \ uses functions, which
are defined by the data types of their inputs and outputs. Local functions are
described by giving their type signature followed by their specification.

\section{Module Decomposition}

The following table is taken directly from the Module Guide document for this project.

\begin{table}[h!]
\centering
\begin{tabular}{p{0.3\textwidth} p{0.6\textwidth}}
\toprule
\textbf{Level 1} & \textbf{Level 2}\\
\midrule

{Hardware-Hiding} & ~ \\
\midrule

\multirow{7}{0.3\textwidth}{Behaviour-Hiding} & Input Parameters\\
& Output Format\\
& Output Verification\\
& Temperature ODEs\\
& Energy Equations\\ 
& Control Module\\
& Specification Parameters Module\\
\midrule

\multirow{3}{0.3\textwidth}{Software Decision} & {Sequence Data Structure}\\
& ODE Solver\\
& Plotting\\
\bottomrule

\end{tabular}
\caption{Module Hierarchy}
\label{TblMH}
\end{table}

\newpage
~\newpage

\section{\hyperref[mHH]{MIS of Hardware-Hiding Module}} \label{M1}

\subsection{Module}  
Hardware-Hiding Module  

\subsection{Uses}  
Provides abstraction for hardware (microphone and speakers), simplifying access for other modules.  

\subsection{Syntax}  

\subsubsection{Exported Constants}  
\begin{itemize}
    \item \texttt{DEFAULT\_MICROPHONE}  
    \item \texttt{DEFAULT\_AUDIO\_OUTPUT}  
\end{itemize}  

\subsubsection{Exported Access Programs}  
\begin{center}  
\begin{tabular}{|p{5cm}|p{2cm}|p{4cm}|p{3cm}|}  
\hline  
\textbf{Name} & \textbf{Input} & \textbf{Output} & \textbf{Exceptions} \\  
\hline  
initializeMicrophone & None & None & InitializationError \\  
initializeAudioOutput & None & None & InitializationError \\  
readMicrophoneBuffer & None & rawAudioData & ReadError \\  
sendToAudioOutput & audioData & None & PlaybackError \\  
\hline  
\end{tabular}  
\end{center}  

\subsection{Semantics}  

\subsubsection{State Variables}  
\begin{itemize}  
    \item \texttt{microphoneState}  
    \item \texttt{audioOutputState}  
\end{itemize}  

\subsubsection{Environment Variables}  
\begin{itemize}  
    \item \texttt{hardwareDriverLibrary}  
    \item \texttt{deviceConfig}  
\end{itemize}  

\subsubsection{Assumptions}  
\begin{itemize}  
    \item Functional hardware to take in user audio (i.e. microphone) and play out audio (i.e. speaker, headphones) is available.
    \item Necessary drivers are installed.  
\end{itemize}  

\subsubsection{Access Routine Semantics}  

\noindent \texttt{initializeMicrophone}():
\begin{itemize}  
    \item \textbf{Transition:} Sets \texttt{microphoneState} to "active".  
    \item \textbf{Exception:} \texttt{InitializationError} if device fails to initialize.  
\end{itemize}  

\noindent \texttt{sendToAudioOutput(audioData)}:
\begin{itemize}  
    \item \textbf{Transition:} Sends \texttt{audioData} to hardware.  
    \item \textbf{Exception:} \texttt{PlaybackError} if playback fails.  
\end{itemize}  

\subsubsection{Local Functions}  
\begin{itemize}  
    \item \texttt{detectAvailableHardware()}  
    \item \texttt{configureDeviceSettings(deviceType)}  
\end{itemize}  

\section{\hyperref[mUI]{MIS of User Interface Module}} \label{M2}  

\subsection{Module}  
User Interface Module  

\subsection{Uses}  
Defines the interaction mechanisms for uploading audio, recording, playback, and accessing generated PDFs.  

\subsection{Syntax}  

\subsubsection{Exported Constants}  
\begin{itemize}
    \item \texttt{DEFAULT\_THEME}  
    \item \texttt{MAX\_UPLOAD\_SIZE}  
\end{itemize}  

\subsubsection{Exported Access Programs}  
\begin{center}  
\begin{tabular}{|p{5cm}|p{2cm}|p{4cm}|p{3cm}|}  
\hline  
\textbf{Name} & \textbf{Input} & \textbf{Output} & \textbf{Exceptions} \\  
\hline  
displayUploadInterface & None & None & RenderError \\  
triggerPlayback & audioData & None & PlaybackError \\  
\hline  
\end{tabular}  
\end{center}  

\subsection{Semantics}  

\subsubsection{State Variables}  
\begin{itemize}  
    \item \texttt{currentScreen}  
    \item \texttt{userPreferences}  
\end{itemize}  

\subsubsection{Environment Variables}  
\begin{itemize}  
    \item \texttt{displayDriver}  
    \item \texttt{inputDevices}  
\end{itemize}  

\subsubsection{Assumptions}  
\begin{itemize}  
    \item User devices support modern UI rendering.  
\end{itemize}  

\subsubsection{Access Routine Semantics}  

\noindent \texttt{displayUploadInterface}():
\begin{itemize}  
    \item \textbf{Transition:} Sets \texttt{currentScreen} to "Upload".  
    \item \textbf{Exception:} \texttt{RenderError} if rendering fails.  
\end{itemize}  

\noindent \texttt{triggerPlayback(audioData)}:
\begin{itemize}  
    \item \textbf{Transition:} Initiates playback of provided \texttt{audioData}.  
    \item \textbf{Exception:} \texttt{PlaybackError} if audio fails to play.  
\end{itemize}  

\subsubsection{Local Functions}  
\begin{itemize}  
    \item \texttt{validateUserInput()}  
    \item \texttt{renderScreen(screenType)}  
\end{itemize}  

\section{\hyperref[mSG]{MIS of Score Generation Module}} \label{M3}  

\subsection{Module}  
Score Generation Module  

\subsection{Uses}  
Generates musical scores in compressed musicXML format (.mxl) from note sequences processed by previous modules.  

\subsection{Syntax}  

\subsubsection{Exported Constants}  
\begin{itemize}
    \item \texttt{DEFAULT\_FONT\_STYLE}  
    \item \texttt{DEFAULT\_PAGE\_SIZE}  
\end{itemize}  

\subsubsection{Exported Access Programs}  
\begin{center}  
\begin{tabular}{|p{5cm}|p{4cm}|p{2cm}|p{3cm}|}  
\hline  
\textbf{Name} & \textbf{Input} & \textbf{Output} & \textbf{Exceptions} \\  
\hline  
generateScore & noteSequence & .mxl file & GenerationError \\  
customizeScoreSettings & settings & None & ValidationError \\  
\hline  
\end{tabular}  
\end{center}  

\subsection{Semantics}  

\subsubsection{State Variables}  
\begin{itemize}  
    \item \texttt{scoreSettings}  
\end{itemize}  

\subsubsection{Environment Variables}  
\begin{itemize}  
    \item \texttt{fileSystemAccess}  
\end{itemize}  

\subsubsection{Assumptions}  
\begin{itemize}  
    \item Input note sequences are formatted correctly.  
    \item The file system is writable for saving .mxl files.  
\end{itemize}  

\subsubsection{Access Routine Semantics}  

\noindent \texttt{generateScore(noteSequence)}:
\begin{itemize}  
    \item \textbf{Output:} An mxl file representing the musical score.  
    \item \textbf{Exception:} \texttt{GenerationError} if input is invalid.  
\end{itemize}  

\noindent \texttt{customizeScoreSettings(settings)}:
\begin{itemize}  
    \item \textbf{Transition:} Updates \texttt{scoreSettings} with new values.  
    \item \textbf{Exception:} \texttt{ValidationError} if settings are invalid.  
\end{itemize}  

\subsubsection{Local Functions}  
\begin{itemize}  
    \item \texttt{validateSettings(settings)}  
    \item \texttt{renderPDF(noteSequence, scoreSettings)}  
\end{itemize}  

\section{\hyperref[mRSM]{MIS of Raw Signal Processing Module}} \label{M4}  

\subsection{Module}  
Raw Signal Processing Module  

\subsection{Uses}  
Processes raw audio signals to prepare them for feature extraction.  

\subsection{Syntax}  

\subsubsection{Exported Constants}  
\begin{itemize}
    \item \texttt{DEFAULT\_SAMPLING\_RATE}  
    \item \texttt{DEFAULT\_FILTER\_SETTINGS}  
\end{itemize}  

\subsubsection{Exported Access Programs}  
\begin{center}  
\begin{tabular}{|p{4cm}|p{3cm}|p{4cm}|p{3cm}|}  
\hline  
\textbf{Name} & \textbf{Input} & \textbf{Output} & \textbf{Exceptions} \\  
\hline  
filterSignal & rawAudioData & filteredAudioData & FilterError \\  
adjustSamplingRate & audioData, rate & resampledAudioData & ResamplingError \\  
\hline  
\end{tabular}  
\end{center}  

\subsection{Semantics}  

\subsubsection{State Variables}  
\begin{itemize}  
    \item \texttt{currentSamplingRate}  
    \item \texttt{filterParameters}  
\end{itemize}  

\subsubsection{Environment Variables}  
\begin{itemize}  
    \item None  
\end{itemize}  

\subsubsection{Assumptions}  
\begin{itemize}  
    \item Input audio data is in a readable format.  
\end{itemize}  

\subsubsection{Access Routine Semantics}  

\noindent \texttt{filterSignal(rawAudioData)}:
\begin{itemize}  
    \item \textbf{Output:} Filters noise and returns cleaned audio data.  
    \item \textbf{Exception:} \texttt{FilterError} if filtering fails.  
\end{itemize}  

\noindent \texttt{adjustSamplingRate(audioData, rate)}:
\begin{itemize}  
    \item \textbf{Output:} Resamples audioData to the desired rate.  
    \item \textbf{Exception:} \texttt{ResamplingError} if resampling fails.  
\end{itemize}  

\subsubsection{Local Functions}  
\begin{itemize}  
    \item \texttt{computeSpectralFeatures(audioData)}
    \item \texttt{applyFilter(rawAudioData, filterParameters)}  
    \item \texttt{resample(audioData, rate)}  
\end{itemize}  

\section{\hyperref[mAFE]{MIS of Audio Feature Extraction Module}} \label{M5}  

\subsection{Module}  
Audio Feature Extraction Module  

\subsection{Uses}  
Extracts meaningful features from processed audio for use in score generation.  

\subsection{Syntax}  

\subsubsection{Exported Constants}  
\begin{itemize}
    \item \texttt{DEFAULT\_FEATURE\_SET}  
    \item \texttt{DEFAULT\_WINDOW\_SIZE}  
\end{itemize}  

\subsubsection{Exported Access Programs}  
\begin{center}  
\begin{tabular}{|p{4.5cm}|p{4cm}|p{2.5cm}|p{3cm}|}  
\hline  
\textbf{Name} & \textbf{Input} & \textbf{Output} & \textbf{Exceptions} \\  
\hline  
extractFeatures & processedAudioData & featureSet & ExtractionError \\  
configureFeatureSettings & settings & None & ValidationError \\  
\hline  
\end{tabular}  
\end{center}  

\subsection{Semantics}  

\subsubsection{State Variables}  
\begin{itemize}  
    \item \texttt{featureSettings}  
\end{itemize}  

\subsubsection{Environment Variables}  
\begin{itemize}  
    \item None  
\end{itemize}  

\subsubsection{Assumptions}  
\begin{itemize}  
    \item Input audio has been processed by the Raw Signal Processing Module.  
\end{itemize}  

\subsubsection{Access Routine Semantics}  

\noindent \texttt{extractFeatures(processedAudioData)}:
\begin{itemize}  
    \item \textbf{Output:} Extracted features such as pitch, tempo, and dynamics.  
    \item \textbf{Exception:} \texttt{ExtractionError} if feature extraction fails.  
\end{itemize}  

\noindent \texttt{configureFeatureSettings(settings)}:
\begin{itemize}  
    \item \textbf{Transition:} Updates \texttt{featureSettings} with new values.  
    \item \textbf{Exception:} \texttt{ValidationError} if settings are invalid.  
\end{itemize}  

\subsubsection{Local Functions}  
\begin{itemize}  
    \item \texttt{calculateTempoDynamics(audioData)}  
\end{itemize}  

\section{\hyperref[mFFC]{MIS of File Format Conversions Module}} \label{M6}  

\subsection{Module}  
File Format Conversions Module
\subsection{Uses}  
Handles conversion of input and output files between supported formats  (.mxl \Rightarrow PDF).  

\subsection{Syntax}  

\subsubsection{Exported Constants}  
\begin{itemize}
    \item \texttt{SUPPORTED\_IMPORT\_FORMATS}  
    \item \texttt{SUPPORTED\_EXPORT\_FORMATS}  
\end{itemize}  

\subsubsection{Exported Access Programs}  
\begin{center}  
\begin{tabular}{|p{3cm}|p{4cm}|p{4cm}|p{3cm}|}  
\hline  
\textbf{Name} & \textbf{Input} & \textbf{Output} & \textbf{Exceptions} \\  
\hline  
importFile & filePath, format & mxl data & ImportError \\  
exportFile & data, format, filePath & None & ExportError \\  
\hline  
\end{tabular}  
\end{center}  

\subsection{Semantics}  

\subsubsection{State Variables}  
\begin{itemize}  
    \item None  
\end{itemize}  

\subsubsection{Environment Variables}  
\begin{itemize}  
    \item \texttt{fileSystemAccess}  
\end{itemize}  

\subsubsection{Assumptions}  
\begin{itemize}  
    \item The specified file path exists for import operations.  
    \item The export destination is writable.  
\end{itemize}  

\subsubsection{Access Routine Semantics}  

\noindent \texttt{importFile(filePath, format)}:
\begin{itemize}  
    \item \textbf{Output:} .mxl data extracted from Raw audio input.  
    \item \textbf{Exception:} \texttt{ImportError} if the file or format is invalid.  
\end{itemize}  

\noindent \texttt{exportFile(data, format, filePath)}:
\begin{itemize}  
    \item \textbf{Output:} Saves data in the specified format at the given file path.  
    \item \textbf{Exception:} \texttt{ExportError} if writing fails.  
\end{itemize}  

\subsubsection{Local Functions}  
\begin{itemize}  
    \item \texttt{convertToRawAudio(fileData, format)}  
    \item \texttt{writeToFile(data, format, filePath)}  
\end{itemize}  

\section{\hyperref[mARP]{MIS of Audio Recording and Playback Module}} \label{M7}  

\subsection{Module}  
Audio Recording and Playback Module  

\subsection{Uses}  
Provides functionalities for recording audio through the microphone and playing back recorded or imported audio.  

\subsection{Syntax}  

\subsubsection{Exported Constants}  
\begin{itemize}
    \item \texttt{DEFAULT\_AUDIO\_FORMAT}  
    \item \texttt{MAX\_RECORDING\_DURATION}  
\end{itemize}  

\subsubsection{Exported Access Programs}  
\begin{center}  
\begin{tabular}{|p{3cm}|p{4cm}|p{4cm}|p{3cm}|}  
\hline  
\textbf{Name} & \textbf{Input} & \textbf{Output} & \textbf{Exceptions} \\  
\hline  
startRecording & None & None & RecordingError \\  
stopRecording & None & rawAudioData & RecordingError \\  
playAudio & audioData & None & PlaybackError \\  
\hline  
\end{tabular}  
\end{center}  

\subsection{Semantics}  

\subsubsection{State Variables}  
\begin{itemize}  
    \item \texttt{isRecording}  
    \item \texttt{currentAudioBuffer}  
\end{itemize}  

\subsubsection{Environment Variables}  
\begin{itemize}  
    \item \texttt{microphoneAccess}  
    \item \texttt{speakerOutput}  
\end{itemize}  

\subsubsection{Assumptions}  
\begin{itemize}  
    \item Microphone and speaker are functional and accessible.  
\end{itemize}  

\subsubsection{Access Routine Semantics}  

\noindent \texttt{startRecording()}:
\begin{itemize}  
    \item \textbf{Transition:} Sets \texttt{isRecording} to true and starts capturing audio from the microphone.  
    \item \textbf{Exception:} \texttt{RecordingError} if the microphone is unavailable.  
\end{itemize}  

\noindent \texttt{stopRecording()}:
\begin{itemize}  
    \item \textbf{Output:} Captured audio as raw data.  
    \item \textbf{Transition:} Sets \texttt{isRecording} to false.  
    \item \textbf{Exception:} \texttt{RecordingError} if no recording is in progress.  
\end{itemize}  

\noindent \texttt{playAudio(audioData)}:
\begin{itemize}  
    \item \textbf{Output:} Plays the specified audio data through the speaker.  
    \item \textbf{Exception:} \texttt{PlaybackError} if playback fails.  
\end{itemize}  

\subsubsection{Local Functions}  
\begin{itemize}  
    \item \texttt{captureMicrophoneInput()}  
    \item \texttt{sendToSpeaker(audioData)}  
\end{itemize}  


\newpage

\bibliographystyle {plainnat}
\bibliography {../../../refs/References}

\newpage

\section{Appendix} \label{Appendix}

\wss{Extra information if required}

\newpage{}

\section*{Appendix --- Reflection}

The information in this section will be used to evaluate the team members on the
graduate attribute of Problem Analysis and Design.

The purpose of reflection questions is to give you a chance to assess your own
learning and that of your group as a whole, and to find ways to improve in the
future. Reflection is an important part of the learning process.  Reflection is
also an essential component of a successful software development process.  

Reflections are most interesting and useful when they're honest, even if the
stories they tell are imperfect. You will be marked based on your depth of
thought and analysis, and not based on the content of the reflections
themselves. Thus, for full marks we encourage you to answer openly and honestly
and to avoid simply writing ``what you think the evaluator wants to hear.''

Please answer the following questions.  Some questions can be answered on the
team level, but where appropriate, each team member should write their own
response:


\begin{enumerate}
  \item What went well while writing this deliverable?  \\ \\
  Emily: \\ \\

  Mark: \\ \\

  Ian: \\ \\

  Jackson: \\ \\
  
  \item What pain points did you experience during this deliverable, and how
    did you resolve them? \\ \\

    Emily: \\ \\

    Mark: \\ \\

    Ian: \\ \\

    Jackson: \\ \\

    
  \item Which of your design decisions stemmed from speaking to your client(s)
  or a proxy (e.g. your peers, stakeholders, potential users)? For those that
  were not, why, and where did they come from? \\ \\

  No changes in design occurred during initial creation of this documentation. Certain design choices were made earlier in the planning stage from discussion with clients \\


  \item While creating the design doc, what parts of your other documents (e.g.
  requirements, hazard analysis, etc), if any, needed to be changed, and why? \\ \\

  N/A \\
  \item What are the limitations of your solution?  Put another way, given
  unlimited resources, what could you do to make the project better? (LO\_ProbSolutions) \\ \\

  Our project aims to create as strong an ability as possible for non-technical musicians to create highly detailed notation, but there are intricacies that are extremely difficult to extract from audio alone. Features like staccato, crescendo, chords, grace notes, or tempo changes are difficult to differentiate from variance that occurs from regular human playing. Tackling this issue effectively would probably best be done with very advanced signal processing and personally trained, or fine-tuned machine learning models. Given more time, it would also be helpful to implement advanced options for users to maximise precision. If for example there is a music piece with lower confidence sections, such an area where there is a similar likelihood of a note being a fast-played 16th note, or a grace note, it could be possible for the user to toggle through most likely interpretations with playback to determine the ideal representation of their playing. \\
  \item Give a brief overview of other design solutions you considered.  What
  are the benefits and tradeoffs of those other designs compared with the chosen
  design?  From all the potential options, why did you select the documented design?
  (LO\_Explores)
\end{enumerate}


\end{document}